\documentclass[11pt]{article}
\usepackage[colorlinks]{hyperref}

\parskip=5pt

\bibliographystyle{alpha}
%\bibliographystyle{plain}
%\bibliographystyle{annotation}

%HOW DO WE GET ALPHA, INCLUDE LOCAL URL AND ANNOTATION?

%EXAMPLES:
%\url{MATERIALS/KEARNS/Kearns-paper-econ+CS.pdf}
%EXAMPLE: \htmladdnormallink{TEXT}{http://hdl.handle.net/1721.1/3719}
%\url{../../READINGS-NEW/Newman-networks-survey-SIAM-2003.pdf}


\def\P{\mathbb{P}}
\begin{document}


\centerline{\bf MASSACHUSETTS INSTITUTE OF TECHNOLOGY}
\par\vskip 4 pt \noindent {6.268} \hfill Spring 2018
\par\noindent Lecture 8: Scale-Free Networks \hfill{3/6/2018}
\par\vskip 5 pt \hrule \par\vskip 10 pt

\vspace{10pt}

\noindent

The lecture will discuss research and controversy around power laws and scale-free networks over the last 20 years. Reading the primary sources beforehand is optional but encouraged. 

\vspace{10pt}
\noindent
{\bf Models of Scale-Free Networks}
\begin{enumerate}
	\item Newman \cite{Newman2010} provides a nice analysis of the models of Price and Barabasi-Albert in Chapter 14. 
	\item An extensive review is provided by Part XI of \cite{Dorogovtsev2002}. 
\end{enumerate}

\vspace{10pt}
\noindent
{\bf The Classic Paper}

\begin{enumerate}
	\item The most famous paper in this area claimed that the World Wide Web, an actor collaboration network, and a power grid network all displayed power-law degree distributions.  Now cited 30,000 times: \cite{Barabasi1999}. 
	\item \cite{barabasi2009} provides a nice illustration of preferential attachment, and a helpful review of developments since the original paper. 
\end{enumerate}

\vspace{10pt}
\noindent
{\bf Measuring Power Laws}
\begin{enumerate}
	\item Many empirical papers plot the degree distribution on log-log axes and find that it appears approximately linear, concluding that it must be a power law. How reliable is this method? Read \cite{Clauset2009} to find out. 
	\item Applying the methodology above, Broido and Clauset published a paper \cite{Broido2017} in January with a startling claim -- ``Scale Free Networks are Rare''! 
\end{enumerate}




\fontsize{10}{10}
\bibliography{/Users/phil/bibs/library.bib}{}

\end{document}

